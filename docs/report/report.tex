\documentclass[12pt]{article}
\usepackage{arxiv}
% \usepackage{hyperref}
\usepackage[colorlinks]{hyperref}
\usepackage[colorinlistoftodos]{todonotes}
\usepackage{xcolor}
\usepackage{soul}
\sethlcolor{yellow}

\title{Training Convolutional Neural Networks on the MNIST Dataset}
\author{Benjamin Talbot \\
	Department of Computer Science\\
	University of New Brunswick Saint John\\
	Saint John, New Brunswick E2K 5E2\\
}
\renewcommand{\headeright}{} % Benjamin Talbot ???
\renewcommand{\undertitle}{CS 4795 Final Project Report}
\renewcommand{\shorttitle}{MNIST Digit Recognition using CNN}
\begin{document}
% \sloppy
\maketitle

\begin{abstract}
	% Important findings from the implementation
	text
\end{abstract}

% \keywords{}
\tableofcontents
\newpage



\section{Introduction}
% Why did you choose an idea to implement
% What did you learn from it
% Machine learning is a



\section{The problem}
% Explain the problem to a 7th-grade student
As we put more and more information on our computers, the need for paper is becoming somewhat obsolete. Even though we still use paper for some things today, it is often useful to have a digital copy of something in addition to the physical copy. This could be for communication purposes, for storage, or for backups, among other reasons. Having machines recognize handwritten documents is an important part of digitizing information. 



\section{Key idea}
% Explain the idea to your friend


\section{The details}
% What is the overall performance
% What were the most striking successes?
% What were the most striking failures? (show pictures if possible)

This project was approached with multiple goals: to learn about convolutional neural networks, to learn various architectures of CNNs, and to use those architectures to train different CNNs on the MNIST dataset. \hl{The performance of the models was} not of much concern, but it was still observed to understand the 
Three experiments were performed during the development of this project. 



\section{Related works}
% What other ideas are there and the reason why you didn't choose them.



\section{Conclusion}
% Summarize your work
% State your contribution (What would be useful for other students)
% Suggest future works



\section{Citations}
\cite{ahlawat2020improved}
\cite{liu2020comparisions}
\cite{pashine2021handwritten}
\cite{wu2017introduction}
\cite{726791}



% Bibliography
% Include the bibliography you found relevant to your project.
% Each bibliography should appear in the body of the final paper.
% \cite{} % cite the MNIST database or the webpage for it?
\bibliographystyle{IEEEtran}
\bibliography{sources}



\newpage



\appendix
\section{Appendix}
% The source code
% The failed data set
%  - the digits that the model did not classify correctly (potentially aggregated, small subsize for picture, maybe show classes with numbers, i.e., histogram like), a discussion about it (like in presentation), and why I think it failed to recognize



\end{document}






















% \usepackage[utf8]{inputenc} % allow utf-8 input
% \usepackage[T1]{fontenc}    % use 8-bit T1 fonts
% \usepackage{booktabs}       % professional-quality tables
% \usepackage{amsfonts}       % blackboard math symbols
% \usepackage{nicefrac}       % compact symbols for 1/2, etc.
% \usepackage{microtype}      % microtypography
% \usepackage{graphicx}
% \usepackage{natbib}
% \usepackage{doi}